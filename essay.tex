% Please do not change the document class
\documentclass{scrartcl}

% Please do not change these packages
\usepackage[hidelinks]{hyperref}
\usepackage[none]{hyphenat}
\usepackage{setspace}
\doublespace

% You may add additional packages here
\usepackage{amsmath}

% Please include a clear, concise, and descriptive title
\title{What are the benefits of having a productive scrum master?}

% Please do not change the subtitle
\subtitle{COMP150 - Agile Development Practice}

% Please put your student number in the author field
\author{1706194}

\begin{document}

\maketitle


\section{Introduction}

I will be discussing the many benefits that come with having a productive scrum master and how that can influence a team, I will also explore any possible downsides that can come with a productive scrum master, I.E, if a scrum master becomes too agitated with the group. I feel like this is important to discuss because it can help improve teams that are currently struggling with appointing a scrum master or if their current scrum master is becoming too hard to work with. It's also important to discuss the importance of a productive scrum master because it can help show good leadership and teamwork. I will also talk about different methods of avoiding any problems that a scrum master can have, as well as any problems a team member could have with the scrum master. 

\section {What is the agile philosophy?}

Agile philosophy is the idea or practice of different development methods, for example, scrum. The agile methods rely on four crucial principles:\\  
"The team" - The team is the most important feature; the team must be able to communicate efficiently to avoid confusion. The team is more important than their tools. 
In the game industry development teams must always be communicating to prevent blocks, change of mechanics or to present ideas or solutions\\ 
"The product" -  The product must always be in working state, regardless if its finished or not. In the game industry, the product should always be ready to be tested, this is to allow bug fixes and test mechanics. Game development teams always have a working alpha build of their game.\\
"Collaboration" -  The development team aren't the only ones responsible for the product, the product owner must always be in communication with the team to provide them information about any changes. In the game industry, the product owner would provide the development team with information regarding due dates, demand, budget, changes to the product.\\ 
"Acceptance of change" the development team must always be ready for change, and be accepting of it, meaning they need to be flexible and always prepared. In the game industry a development team might have to change certain mechanics that fit the game better, or change something within the game narrative so it flows smoother. The developers might have to push the game out before their due date, cutting certain elements of the game.\\     
"Agile User-Centered Method by Dominique Deuff , and Mathilde Cosquer CHAPTER 2. introduction to the methods employed" 

\section {The scrum Master and scrum}\\  
"Professional ScrumMasters handbook by Stacia Viscradi - 2013 chapter 1" “A well-balanced, effective ScrumMaster understands both the principles and the practices of scrum” this means that if a productive ScrumMaster is efficient and effective at their role, then they will be able to apply their knowledge of the scrum method to help keep their team be more organised and flexible, ready for any changes they might encounter. The game industry goes this by having an experienced ScrumMaster that has great management and leadership skills.\\  
"The agile pocket guide by Peter saddington - 2012 Chapter 6" “Your agile team is more effective than any other tribe out in the jungle because your team members constantly communicate with one another” This reference means that if the scrum master isn't communicating with the team, or isn't pushing the team to communicate, the team won't be able to work as efficiently. The reason why Scrum is such an affective agile method, is because the of the communication within the team. The ScrumMaster is responsible for organising regular meetings called "stand ups" which allow the team to briefly discuss what they have accomplished, what they will start to work on, and if there are any blocks in the team. A ScrumMaster within the game development team would use the "stand ups" to ask the team what progress they have made and if any changes need to be made.\\   
"Software in 30 days by Ken Schwaber, Jeff Sutherland - 2012 Appendix 2 page 136" “Scrum is a framework for developing and sustaining complex products” this means that if the scrum master is productive and efficient, they can help the team complete the complex task that are needed for the products they are making. A good ScrumMaster would use programs such as a Trellobaord which is used to break down different aspects of the product to be worked on. Almost like a more effective to do list. They use these tools to create smaller tasks for induvial to complete. The ScrumMaster would use these techniques within the game industry by giving certain departments different task, like creating basic game assets for the art developers or making the character preform basic move functions for the programming developers\\
\section{Conclusion}

The ScrumMaster is a very important role for the Scrum agile method. They are in charge of organising daily meets, keeping the team flexible and ready for changes, making sure the team are on the same page and communicating and making sure the team is up to date with the product owner and their needs. This means the more productive the ScrumMaster, the easier it is for the team to finish their task and get the product released on time. If the ScrumMaster wasn’t as productive as they need to be, then the team would crumble and fail as there would be lack of communication and direction. 

\bibliographystyle{ieeetran}
\bibliography{references}

\end{document}
